\pagestyle{empty}

\noindent \textbf{\Large CD Használati útmutató}

\vskip 1cm

A CD tartalma:
\begin{itemize}
\item a dolgozatot egy \texttt{dolgozat.pdf} fájl formájában,
\item a LaTeX forráskódja a dolgozatnak,
\item az elkészített program forrás projekt,
\item az elkészített program (exe),
\item példa OpenCl C nyelvű programok,
\item egy útmutató saját instrukciók létrehozására.
\end{itemize}

Ajánlott a program fordítása, ehhez a következő keretrendszerek és könvytárak szükségesek: 
\begin{itemize}
\item DotNet 6.0,
\item Blazor.Extensions.Canvas NuGet csomag,
\item Newtonsoft.Json NuGet csomag,
\item System.Drawing.Common NuGet csomag,
\item System.Management  NuGet csomag.
\end{itemize}

Nem kötelező, de ajánlott : Visual Studio.
A fordítás ajánlott módja Windows-on a projekt megnyitása (Szakdolgozat.csproj) Visual Studio-ban, majd a Szakdolgozat gombra kattintva automatikusan lefut a fordítása folyamat és elindul a program, és egy hozzá tartozó webszerver, alaphelyzetben IIS Express, valamint egy böngészőben a kezdőoldal.
Ha ez nem megoldható, akkor lehet a projekt mappában futtatni a "dotnet build" és a "dotnet run" parancsokat, amik lefordítják, illetve futtatják a programot. Ha nem nyílna meg a böngésző, akkor elérjük a webalkalmazást a parancssorban megjelent localhost címen.

Ha a már elkészített programot szeretnénk futtatni, megtehetjük az executable elindításával, de ekkor nekünk kell webszerverről gondoskodni.

%Ennek a címe lehet például \textit{A mellékelt CD tartalma} vagy \textit{Adathordozó használati útmutató} is.
%
%Ez jellemzően csak egy fél-egy oldalas leírás.
%Arra szolgál, hogy ha valaki kézhez kapja a szakdolgozathoz tartozó CD-t, akkor tudja, hogy mi hol van rajta.
%Jellemzően elég csak felsorolni, hogy milyen jegyzékek vannak, és azokban mi található.
%Az elkészített programok telepítéséhez, futtatásához tartozó instrukciók kerülhetnek ide.
%
%A CD lemezre mindenképpen rá kell tenni
%\begin{itemize}
%\item a dolgozatot egy \texttt{dolgozat.pdf} fájl formájában,
%\item a LaTeX forráskódját a dolgozatnak,
%\item az elkészített programot, fontosabb futási eredményeket (például ha kép a kimenet),
%\item egy útmutatót a CD használatához (ami lehet ez a fejezet külön PDF-be vagy MarkDown fájlként kimentve).
%\end{itemize}
