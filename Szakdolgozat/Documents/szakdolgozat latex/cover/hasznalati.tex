\pagestyle{empty}

\noindent \textbf{\Large CD Használati útmutató}

\vskip 1cm

A CD tartalma:
\begin{itemize}
\item a dolgozatot egy \texttt{dolgozat.pdf} fájl formájában,
\item a \LaTeX\enskip forráskódja a dolgozatnak,
\item az elkészített program forrás projekt,
\item az elkészített program (\textit{exe}),
\item példa OpenCL C nyelvű programok,
\item egy útmutató saját instrukciók létrehozására.
\end{itemize}

Ajánlott a program fordítása. Ehhez a következő keretrendszerek és könvytárak szükségesek: 
\begin{itemize}
\item \textit{DotNet 6.0},
\item \textit{Blazor.Extensions.Canvas NuGet} csomag,
\item \textit{Newtonsoft.Json NuGet} csomag,
\item \textit{System.Drawing.Common NuGet} csomag,
\item \textit{System.Management  NuGet} csomag.
\end{itemize}

Nem kötelező, de ajánlott: Visual Studio.
A fordítás ajánlott módja Windows-on a solution megnyitása (\textit{Szakdolgozat.sln}) Visual Studio-ban, majd a \textit{Szakdolgozat} gombra kattintva automatikusan lefut a fordítási folyamat, elindul a program, és egy hozzá tartozó webszerver, alaphelyzetben IIS Express, valamint egy böngészőben a kezdőoldal.
Ha ez nem megoldható, akkor lehet a projekt mappában futtatni a \textit{"dotnet build"} és a \textit{"dotnet run"} parancsokat, amik lefordítják, illetve futtatják a programot. Ha nem nyílna meg a böngésző, akkor elérjük a webalkalmazást a parancssorban megjelent \textit{localhost} címen.

Ha a már elkészített programot szeretnénk futtatni, megtehetjük az executable elindításával, de ekkor nekünk kell webszerverről gondoskodni.

%Ennek a címe lehet például \textit{A mellékelt CD tartalma} vagy \textit{Adathordozó használati útmutató} is.
%
%Ez jellemzően csak egy fél-egy oldalas leírás.
%Arra szolgál, hogy ha valaki kézhez kapja a szakdolgozathoz tartozó CD-t, akkor tudja, hogy mi hol van rajta.
%Jellemzően elég csak felsorolni, hogy milyen jegyzékek vannak, és azokban mi található.
%Az elkészített programok telepítéséhez, futtatásához tartozó instrukciók kerülhetnek ide.
%
%A CD lemezre mindenképpen rá kell tenni
%\begin{itemize}
%\item a dolgozatot egy \texttt{dolgozat.pdf} fájl formájában,
%\item a LaTeX forráskódját a dolgozatnak,
%\item az elkészített programot, fontosabb futási eredményeket (például ha kép a kimenet),
%\item egy útmutatót a CD használatához (ami lehet ez a fejezet külön PDF-be vagy MarkDown fájlként kimentve).
%\end{itemize}
