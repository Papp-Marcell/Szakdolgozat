\Chapter{Bevezetés}

A nagy teljesítményű számítástechnika, vagyis a High Performance Computing kiemelkedő szerepet tölt be a modern informatikában. Az egyik első szuperszámítógépet 1960-ban építette az UNIVAC \cite{larc}, és azóta óriási fejlődésen ment át ez a terület. Erre a fejlődésre pedig szükség van, hogy a erőforrás-igényes feladatok emberi időn belül végbemenjenek. A feladatok között találunk példákat quantum mechanikákra, időjárás előrejelzésre, klímaváltozás szimulálására, olajkutatásra, rákkutatásra, molekula modellezésre, nukleáris fúzió elemzésére és sok másra. Kutatók hamar rájöttek, hogy a feladatok nagy része jobban, gyorsabban megoldható lenne párhuzamos módszerekkel, és ezért párhuzamos programozási megoldásokra van szükség.

Régebben a legtöbb párhuzamos technológia a CPU-ra alapult és figyelmen kívül hagyta a GPU-t ami rendeltetés szerint nagyon sok, de kisebb számítás elvégzésére alkalmas. Ennek a hiánynak a betöltésére két technológia terjedt el, az egyik az NVIDIA által létrehozott és fenntartott CUDA \cite{cuda}, és a nyílt szabványú OpenCL \cite{opencl}. A CUDA sajnálatos módon limitálva van az NVIDIA által gyártott GPU-k-ra, az OpenCL viszont egy nyílt keretrendszer, amivel változatos platformokon futni képes programokat lehet létrehozni.

 A dolgozat célja, hogy betekintést nyújtson az OpenCL programok működésébe. Ennek érdekében bemutatja magát a szabványt, példaprogramokat, valamint egy egyedi elemző rendszert, ami a program futását hivatott szemléltetni.

Egy párhuzamos program teljesítményének kiértékelése, vagy működésének megértése kihívást jelenthet. Hogy segíteni tudjuk a fejlesztő munkáját, létrehozunk egy elemző alkalmazást, ami útmutatást és visszajelzést ad számára. Mivel egy nagyobb program több milliárd különböző számítást is elvégezhet másodpercenként, egy szövegalapú kiértékelő rendszer nehezen átlátható lehet. Ebből kifolyólag szükség van egy vizuális megjelenítő részre, ami könnyen értelmezhető. Az elemző rendszer többféle módon tudja nyomon követni a párhuzamos alkalmazás futását. A legkézenfekvőbb megoldás talán a tényleges futás közbeni elemzés, a használt erőforrások nyomon követése. Talán kicsit egyszerűbb ennél a program szimulálása, vagy emulálása, vagy a program kódjának elemzése, és az alapján egy modell felállítása.
