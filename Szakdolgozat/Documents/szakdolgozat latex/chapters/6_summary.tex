\Chapter{Összefoglalás}

A szakdolgozat készítése során tanulmányozva voltak különböző, párhuzamos számítást megvalósító módszerek, szabványok. Ezek közül a hangsúly az OpenCL szabványon volt, aminek működése mélyrehatóan vizsgálva és tanulmányozva volt. Ezen kívül megismertem több webalkalmazással kapcsolatos technológiát, majd egy átfogó keretrendszerben a megszerzett OpenCL ismereteimet felhasználva elkezdtem a megtervezett program implementációját. A szimuláció megalkotásához nagy segítségre voltak a fordítókkal és az átmeneti kóddal kapcsolatban megszerzett ismereteim. \\

A dolgozat keretében bemutatásra került az OpenCL szabvány, története, elemei, sajátosságai.
Elkészült egy platform független alkalmazás, ami ötvözi a .Net 6.0 és a Blazor webalkalmazás keretrendszereket, valamint több más naprakész technológiát. Bemtatásra került a program tervezése, elkészítése, és működése. A program sikeresen bemutatja a használt erőforrásokat, az OpenCL működését az elemzett forrásfájlban, valamint pontos szimulációt végez a program aritmetikája alapján.\\

Egyetemi tanulmányaim során egy tárgyam se tért ki részletesebben se az OpenCL-re, de főleg az összetett webalkalmazásokra. Viszont más oktatott párhuzamos technológiák, mint az MPI előzetes ismerete, nagyban segítette a dolgozat alakulását. A webalkalmazás viszont főleg a dolgozat készítése során megszerzett tapasztalatokból épült fel, mivel az egyetemen bemutatott JavaScript és Angular\cite{angular} alkalmazás nem volt a legalkalmasabb erre a projektre. \\

A tervezettnél kevésbé sikerült a program szimulációja, mivel a forrásfájlból átmeneti kód készítése nem bizonyult a dolgozat keretein belül reálisan megoldhatónak, ezért az átmeneti kódot más forrásból kellett a szimulációba injektálni. Mégis a szimulációt nevezném a dolgozat egyik legérdekesebb részének, főleg a dinamikus osztály létrehozása és a párhuzamosság szimulációja miatt.\\

A feladatkiírásban részletezett pontok megvalósítására és bemutatására sor került, de akadnak még lehetőségek a bővítésre és a tökéletesítésre. További fejlesztési lehetőség marad az átmeneti kód direkten a forrásfájlból elkészítése, azaz egy igazi fordító létrehozása. Fejleszthető még a felhasználói felület és a grafikai megjelenítés is.\\

%Hasonló szerepe van, mint a bevezetésnek.
%Itt már múltidőben lehet beszélni.
%A szerző saját meglátása szerint kell összegezni és értékelni a dolgozat fontosabb eredményeit.
%Meg lehet benne említeni, hogy mi az ami jobban, mi az ami kevésbé jobban sikerült a tervezettnél.
%El lehet benne mondani, hogy milyen további tervek, fejlesztési lehetőségek vannak még a témával kapcsolatban.
