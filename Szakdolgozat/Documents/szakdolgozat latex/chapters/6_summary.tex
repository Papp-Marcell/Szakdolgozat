\Chapter{Összefoglalás}

A dolgozat keretében bemutatásra került az OpenCl szabvány, története, elemei, sajátosságai.
Elkészült egy platform független alkalmazás, ami ötvözi a .Net 6.0 és a Blazor webalkalmazás keretrendszereket, valamint több más naprakész technológiát. Be lett mutatva a program designja, elkészítése, és működése. A program sikeresen bemutatja a használt erőforrásokat, az OpenCl működését az elemzett forrásfájlban, valamint pontos szimulációt végez a program aritmetikája alapján.

A tervezettnél kevésbé sikerült a program szimulációja, mivel a forrásfájlból átmeneti kód készítése meghaladta a dolgozatot, az átmeneti kódot más forrásból kellett a szimulációba injektálni.

További fejlesztési lehetőség  marad az átmeneti kód direkten a forrásfájlból elkészítése, azaz egy igazi fordító létrehozása. Fejleszthető még a felhasználói felület és a grafikai megjelenítés.

%Hasonló szerepe van, mint a bevezetésnek.
%Itt már múltidőben lehet beszélni.
%A szerző saját meglátása szerint kell összegezni és értékelni a dolgozat fontosabb eredményeit.
%Meg lehet benne említeni, hogy mi az ami jobban, mi az ami kevésbé jobban sikerült a tervezettnél.
%El lehet benne mondani, hogy milyen további tervek, fejlesztési lehetőségek vannak még a témával kapcsolatban.
